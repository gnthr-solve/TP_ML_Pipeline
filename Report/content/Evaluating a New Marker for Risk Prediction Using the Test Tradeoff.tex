
\section{Evaluating a New Marker for Risk Prediction Using the Test Tradeoff}

\subsection{Definitions (adapted to previous notation)}
	Let $\rho$ be a risk score a model outputs, where $\rho(X)$ is the score for instance $X$.
	Let $D\in \{0,1\}$ be the r.v. that states disease $D=1$ or not disease $D=0$.
	Then probability of developing disease if the risk score is $\rho(X)=s$ is given by
	\[
		R_s = \mathbb{P}(D=1 | \rho(X) = s)
	\]
	Requirement: If $s_1 < s_2 \Rightarrow R_{s_1} < R_{s_2}$
	Cutoff $\tau$ defined such that 
	\[
	\begin{aligned}
    		\rho(X) \geq \tau &\Rightarrow \hat{Y} = 1 \\
    		\rho(X) < \tau &\Rightarrow \hat{Y} = 0 \\
    	\end{aligned}
	\]
	Then the True Positive Rate and the False Positive Rate are
	\[
	\begin{aligned}
    		TPR(\tau) &= \mathbb{P}(\rho(X) \geq \tau | D = 1) \\
    		FPR(\tau) &= \mathbb{P}(\rho(X) \geq \tau | D = 0) \\
    	\end{aligned}
	\]
	We say the probability of having/developing the disease is $\mathbb{P}(D=1)$
	Let $T$ be treatment or no treatment and define the utilities
	\[
	\begin{aligned}
    		U_{T, D} 			&\quad \text{utility of treating person with disease}\\
		U_{\neg T, D} 		&\quad \text{utility of not treating person with disease}\\
		U_{\neg T, \neg D} 	&\quad \text{utility of not treating person without disease}\\
		U_{T, \neg D} 		&\quad \text{utility of treating person without disease}\\
    	\end{aligned}
	\]
	and $U_{test}$ the negative utility / harm of a test.
	
\subsection{Assumptions}

	Assumptions:
	\[
	\begin{aligned}
    		U_{T, D} 	&> U_{\neg T, D}\\
		U_{\neg T, \neg D} &> U_{T, \neg D} \\
		U_{test} &< 0 \\
    	\end{aligned}
	\]
	
\subsection{Comparison of the Maximum Expected Utility of Risk Prediction}	

	Comparison of the net benefit of Model 2 versus Model 1 is a comparison of the maximum expected utility of risk prediction under Model 2 versus Model 1.
	The maximum expected utility of risk prediction is the maximum, over the cutpoints, of the expected utility of risk prediction.
	The \textbf{expected utility of risk prediction} is
	\[
	\begin{aligned}
    		U_{pred}(\tau) 
		=& P \mathbb{P}(\rho(X) \geq \tau | D = 1) U_{T, D}\\
    		&+ P \mathbb{P}(\rho(X) < \tau | D = 1) U_{\neg T, D}\\
		&+ (1 - P) \mathbb{P}(\rho(X) \geq \tau | D = 0) U_{T, \neg D}\\
		&+ (1 - P) \mathbb{P}(\rho(X) < \tau | D = 0) U_{\neg T, \neg D}\\
		&+ U_{test}
    	\end{aligned}
	\]
	These definitions allow to express 
	\textbf{Fundamental Rule Version 1:}
	\[
		\text{select model 2 if}: \quad \max_\tau U^2_{pred}(\tau) > \max_\tau U^1_{pred}(\tau)
	\]
	
	Comparing maximum expected utilities of risk prediction: a simplification involving the no treatment option
	
	Setting the probabilities for no treatment to $1$ and the ones for treatment to $0$ the expected utility of no treatment at all is:
	\[
		U_{\neg T} =  P  U_{\neg T, D}+ (1 - P)  U_{\neg T, \neg D}
	\]
	and analogously the expected utility of treating everyone is
	\[
		U_{ T} =  P  U_{T, D} + (1 - P)  U_{T, \neg D}
	\]
	The expected utility of risk prediction in excess of the expected utility of no treatment is thus
	\[
	\begin{aligned}
		U_{pred*}(\tau) =& U_{pred}(\tau) - U_{\neg T}
		=& P \mathbb{P}(\rho(X) \geq \tau | D = 1) U_{T, D}\\
    		&+ P \mathbb{P}(\rho(X) < \tau | D = 1) U_{\neg T, D}\\
		&+ (1 - P) \mathbb{P}(\rho(X) \geq \tau | D = 0) U_{T, \neg D}\\
		&+ (1 - P) \mathbb{P}(\rho(X) < \tau | D = 0) U_{\neg T, \neg D}\\
		&+ U_{test} \\
		&- P  U_{\neg T, D} - (1 - P)  U_{\neg T, \neg D} \\
		=& P \mathbb{P}(\rho(X) \geq \tau | D = 1) U_{T, D}\\
		&- P \mathbb{P}(\rho(X) \geq \tau | D = 1) U_{\neg T, D}\\
		&+ (1 - P) \mathbb{P}(\rho(X) \geq \tau | D = 0) U_{T, \neg D}\\
		&- (1 - P) \mathbb{P}(\rho(X) \geq \tau | D = 0) U_{\neg T, \neg D}\\
		&+ U_{test} \\
		=& P \mathbb{P}(\rho(X) \geq \tau | D = 1) (U_{T, D} - U_{\neg T, D})\\\
		&+ (1 - P) \mathbb{P}(\rho(X) \geq \tau | D = 0) (U_{T, \neg D} - U_{\neg T, \neg D})\\
		&+ U_{test} \\
	\end{aligned}
	\]
	When defining
	\[
	\begin{aligned}
		B :=&  U_{T, D} - U_{\neg T, D} \\
		C :=&  U_{T, \neg D} - U_{\neg T, \neg D} \\
	\end{aligned}
	\]
	we can write this as
	
	\[
		U_{pred*}(\tau) = P \mathbb{P}(\rho(X) \geq \tau | D = 1) B + (1 - P) \mathbb{P}(\rho(X) \geq \tau | D = 0) C + U_{test}
	\]
	
	The \textbf{risk threshold}, denoted $T$, is the probability of developing disease in the population at which the expected utility of treatment and no treatment is the same.
	We obtain it by substituting $T$ for $P$ when setting $U_{\neg T} = U_{T}$ in equations and solving for $T$ to get the following formula for the risk threshold, 
	\[
		T =  \frac{C}{C + B}
	\]
	and for the odds of the risk threshold this gives
	\[
		 \frac{T}{1 - T} =  \frac{C}{B}
	\]


\subsection{Net Benefit}
	The \textbf{Net Benefit} $NB(\tau)$ is defined as 
	
	\[
	\begin{aligned}
		NB(\tau) =& \frac{U_{pred}(\tau) - U_{\neg T}}{B} \\
		=& \frac{P \mathbb{P}(\rho(X) \geq \tau | D = 1) B + (1 - P) \mathbb{P}(\rho(X) \geq \tau | D = 0) C + U_{test}}{B} \\
		=& P \mathbb{P}(\rho(X) \geq \tau | D = 1) + (1 - P) \mathbb{P}(\rho(X) \geq \tau | D = 0) \frac{C}{B} + \frac{U_{test} }{B} \\
		& \text{substituting the risk threshold} \\
		=& P \mathbb{P}(\rho(X) \geq \tau | D = 1) + (1 - P) \mathbb{P}(\rho(X) \geq \tau | D = 0) \frac{T}{1 - T} + \frac{U_{test} }{B} \\
	\end{aligned}
	\]
	
	We can also see that, with $P = \mathbb{P}(D=1)$ we can replace the conditional with a joint probability
	\[
	\begin{aligned}
		NB(\tau) =& P \mathbb{P}(\rho(X) \geq \tau | D = 1) + (1 - P) \mathbb{P}(\rho(X) \geq \tau | D = 0) \frac{T}{1 - T} + \frac{U_{test} }{B} \\
		=& \mathbb{P}(\rho(X) \geq \tau , D = 1) + \mathbb{P}(\rho(X) \geq \tau , D = 0) \frac{T}{1 - T} + \frac{U_{test} }{B} \\
		& \text{for concrete model the joint probabilities represent the TP and FP respectively} \\
		=& \frac{\text{TP}}{N} + \frac{T}{1 - T} \frac{\text{FP}}{N} + \frac{U_{test} }{B} \\
	\end{aligned}
	\]
	So in absence of testing costs we have:
	\[
		NB(\tau) = \frac{\text{TP} + \frac{T}{1 - T} \text{FP}}{N}
	\]
	This net benefit for a decision curve, $NB$, is the maximum benefit of risk prediction (in excess of the benefit of no treatment)
	in units of the benefit of treating a true positive. 
	It equals the benefit of treating a true positive after subtracting the cost of treating a false positive at an “exchange rate” based on the risk threshold.
	
	
	
	Optimization Requirement, $R_t = T$,
	$ROCSLOPEt = \frac{1 - P}{P}\frac{T}{1 - T}$
	
	
	
	
	
	
	