\section{DECISION-ANALYTIC EVALUATION OF MARKERS AND MODELS}
	An important aspect of prediction models is the extent to which estimated risks are calibrated (i.e., correspond to observed outcomes). 
	For example, 1 in 20 women with an estimated risk of $5\%$ should have the disease. 
	Calibration problems may especially occur in external validation studies, where systematic differences between predicted risks and observed outcomes are often found. 
	If the model is not well calibrated, some patients will not be managed as intended. 
	Such miscalibration also affects model performance using decision-analytic measures as w becomes inconsistent with TP and FP.
	
	